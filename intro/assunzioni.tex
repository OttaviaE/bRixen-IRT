% Options for packages loaded elsewhere
\PassOptionsToPackage{unicode}{hyperref}
\PassOptionsToPackage{hyphens}{url}
%
\documentclass[
  ignorenonframetext,
]{beamer}
\usepackage{pgfpages}
\setbeamertemplate{caption}[numbered]
\setbeamertemplate{caption label separator}{: }
\setbeamercolor{caption name}{fg=normal text.fg}
\beamertemplatenavigationsymbolsempty
% Prevent slide breaks in the middle of a paragraph
\widowpenalties 1 10000
\raggedbottom
\setbeamertemplate{part page}{
  \centering
  \begin{beamercolorbox}[sep=16pt,center]{part title}
    \usebeamerfont{part title}\insertpart\par
  \end{beamercolorbox}
}
\setbeamertemplate{section page}{
  \centering
  \begin{beamercolorbox}[sep=12pt,center]{part title}
    \usebeamerfont{section title}\insertsection\par
  \end{beamercolorbox}
}
\setbeamertemplate{subsection page}{
  \centering
  \begin{beamercolorbox}[sep=8pt,center]{part title}
    \usebeamerfont{subsection title}\insertsubsection\par
  \end{beamercolorbox}
}
\AtBeginPart{
  \frame{\partpage}
}
\AtBeginSection{
  \ifbibliography
  \else
    \frame{\sectionpage}
  \fi
}
\AtBeginSubsection{
  \frame{\subsectionpage}
}
\usepackage{amsmath,amssymb}
\usepackage{lmodern}
\usepackage{iftex}
\ifPDFTeX
  \usepackage[T1]{fontenc}
  \usepackage[utf8]{inputenc}
  \usepackage{textcomp} % provide euro and other symbols
\else % if luatex or xetex
  \usepackage{unicode-math}
  \defaultfontfeatures{Scale=MatchLowercase}
  \defaultfontfeatures[\rmfamily]{Ligatures=TeX,Scale=1}
\fi
\usetheme[]{metropolis}
% Use upquote if available, for straight quotes in verbatim environments
\IfFileExists{upquote.sty}{\usepackage{upquote}}{}
\IfFileExists{microtype.sty}{% use microtype if available
  \usepackage[]{microtype}
  \UseMicrotypeSet[protrusion]{basicmath} % disable protrusion for tt fonts
}{}
\makeatletter
\@ifundefined{KOMAClassName}{% if non-KOMA class
  \IfFileExists{parskip.sty}{%
    \usepackage{parskip}
  }{% else
    \setlength{\parindent}{0pt}
    \setlength{\parskip}{6pt plus 2pt minus 1pt}}
}{% if KOMA class
  \KOMAoptions{parskip=half}}
\makeatother
\usepackage{xcolor}
\newif\ifbibliography
\setlength{\emergencystretch}{3em} % prevent overfull lines
\providecommand{\tightlist}{%
  \setlength{\itemsep}{0pt}\setlength{\parskip}{0pt}}
\setcounter{secnumdepth}{-\maxdimen} % remove section numbering
\usepackage{graphicx}
\usepackage{setspace}
\usepackage{tabularx}
\usepackage[english]{babel}
\usepackage{tikzsymbols}
\usepackage{subcaption}
\usepackage{tikz}
\usepackage[absolute,overlay]{textpos}
\usepackage{spot}
\usepackage{tabularx}
\usepackage[absolute,overlay]{textpos}
\usepackage{booktabs}
\setbeamertemplate{navigation symbols}{}
\newcommand\Factor{1.2}
\setbeamerfont{subtitle}{size=\large, series=\bfseries}
\useinnertheme{circles}
\useoutertheme{tree}
\definecolor{latenti}{RGB}{54, 114, 89}
\definecolor{manifeste}{RGB}{179, 7, 27}
\definecolor{background}{RGB}{251, 251, 251}
\definecolor{highlight}{RGB}{18, 10, 143}
\setbeamercolor{frametitle}{bg=background}
\setbeamertemplate{frametitle}[default][center]
\AtBeginSubsection{\frame{\subsectionpage}}
\setbeamercolor{section name}{fg=white}
\setbeamersize{text margin left=5mm,text margin right=5mm}
\def\tikzoverlay{\tikz[remember picture, overlay]\node[every overlay node]}
\definecolor{person}{RGB}{33, 113, 181}
\definecolor{question}{RGB}{181, 102, 33}
\newcommand*{\sbj}[1]{\textcolor{person}{#1}}
\newcommand*{\colit}[1]{\textcolor{question}{#1}}
\newcommand*{\high}[1]{\textcolor{highlight}{#1}}
\ifLuaTeX
  \usepackage{selnolig}  % disable illegal ligatures
\fi
\IfFileExists{bookmark.sty}{\usepackage{bookmark}}{\usepackage{hyperref}}
\IfFileExists{xurl.sty}{\usepackage{xurl}}{} % add URL line breaks if available
\urlstyle{same} % disable monospaced font for URLs
\hypersetup{
  pdftitle={Item Response Theory for beginners},
  pdfauthor={Dr.~Ottavia M. Epifania},
  hidelinks,
  pdfcreator={LaTeX via pandoc}}

\title{Item Response Theory for beginners}
\subtitle{Assuzioni e stima dei parametri}
\author{Dr.~Ottavia M. Epifania}
\date{Corso IRT @ Università Libera di Bolzano, ~17-18 Gennaio 2023}
\institute{Bressanone}

\begin{document}
\frame{\titlepage}

\begin{frame}[allowframebreaks]
  \tableofcontents[hideallsubsections]
\end{frame}
\hypertarget{assunzioni}{%
\section{Assunzioni}\label{assunzioni}}

\hypertarget{unidimensionalituxe0}{%
\subsection{Unidimensionalità}\label{unidimensionalituxe0}}

\begin{frame}{}
\protect\hypertarget{section}{}
L'assunzione di \textcolor{highlight}{\textbf{unidimensionalità}} indica
che un solo tratto latente è responsabile delle risposte agli item

Tale assunzione viene spesso valutata mediante modelli di
\textcolor{highlight}{analisi fattoriale confermativa}

I modelli IRT multidimensionali consentono di gestire la presenza di più
tratti latenti ma sono modelli più complessi e meno diffusi

\begin{block}{In caso di violazione}

\vspace{3mm}
Il modello IRT scelto può essere applicato, ma: 

- si possono ottenere stime dei parametri distorte/non intepretabili

- anche se si ottenessero delle stime interpretabili, queste non hanno senso perché il modello non ha senso

\end{block}
\end{frame}

\begin{frame}{Come si valuta}
\protect\hypertarget{come-si-valuta}{}
Solitamente, si utilizza l'analisi fattoriale confermativa
\textcolor{highlight}{(CFA)}.

Se la soluzione ad un fattore presenta una buona fit, si suppone
l'unidimensionalità

\begin{block}{Indici di fit CFA}

\vspace{3mm}
Comparative Fit Index (CFI) $> .90$


Standardized Root Mean Square Residual (SRMSR) $<.08$

Root Mean Sqaure Error of Approximation (RMSEA) $<. 08$

\end{block}
\end{frame}

\hypertarget{indipedenza-locale}{%
\subsection{Indipedenza locale}\label{indipedenza-locale}}

\begin{frame}{}
\protect\hypertarget{section-1}{}
L'assunzione di \textcolor{highlight}{\textbf{indipendenza locale}}
indica che non esiste alcuna relazione tra le risposte di un soggetto ad
item diversi dopo aver controllato per il tratto latente

Le risposte di un soggetto a un insieme di item sono indipendenti quando
la probabilità associata alle risposte fornite dal soggetto agli item è
uguale al prodotto delle probabilità relative alle singole risposte

\begin{block}{In caso di violazione}


\vspace{3mm}
Si rischia di sovrastimare l’informatività (attendibilità) del test
\end{block}
\end{frame}

\begin{frame}{Come si valuta}
\protect\hypertarget{come-si-valuta-1}{}
Si correlano i \textcolor{highlight}{\textbf{residui}} (i.e., differenza
tra la risposta data da un soggetto ad un item e il valore atteso per
quella risposta)

La correlazione tra i residui si interpreta attraverso la statistica
\(Q3\) (Yen, 1984)

In genere, \(Q3\geq .20\) per una coppia di item è indicativo di
dipendenza locale (sono disponibili anche altri cut-off)
\end{frame}

\begin{frame}{Cosa fare}
\protect\hypertarget{cosa-fare}{}
Eliminare uno dei due item della coppia (solitamentem quello con minore
discriminatività o che ha una fit peggiore)

Se si utilizza una procedura adattiva (e.g., Computerized Adaptive
Testing) \(\rightarrow\) vincolare la somministrazione di uno solo dei
due item

Se possibile \(\rightarrow\) combinare i due item in unico item
\includegraphics[width=.10\linewidth]{"img/frank.jpg"}
\end{frame}

\hypertarget{monotonicituxe0}{%
\subsection{Monotonicità}\label{monotonicituxe0}}

\begin{frame}{}
\protect\hypertarget{section-2}{}
\textcolor{highlight}{\textbf{Monotonicità}}: La probabilità di
rispondere correttamente aumenta all'aumentare del livello di tratto
latente

\begin{block}{Come si valuta}

\vspace{3mm}
Si valuta per ogni item

Coefficiente $H$ di Mokken (Originariamente di Loevinger). 

$H \geq .3$: Item accettabile 

$H \geq .5$: Item eccellente 

$H < .3$: Item non accettabile 


\end{block}

\begin{block}{In caso di violazione}

\vspace{3mm}
Effetti negativi sull'attendibilità e validità della scala
\end{block}
\end{frame}

\hypertarget{fit-del-modello}{%
\section{Fit del modello}\label{fit-del-modello}}

\begin{frame}{}
\protect\hypertarget{section-3}{}
Dopo aver verificato le assunzioni, si può procedere alla verifica della
fit del modello mediante due statistiche principali:

\begin{itemize}
\item
  \textcolor{highlight}{\textbf{$M^2$}} (Meyedeu-olivares \& Joe, 2005):
  si basa sulla classificazione dei soggetti in base al loro pattern di
  risposta. Si basa sulla distribuzione \(\chi^2\) per cui tende ad
  essere significativo per campioni ampi (anche se il modello ha una
  buona fit)
\item
  \textcolor{highlight}{\textbf{Root Mean Square Error of Approximation}}
  (\textcolor{highlight}{\textbf{RMSEA}}): Misura di quanto il modello
  si avvicina alla realtà:

  \begin{itemize}
  \tightlist
  \item
    \(\leq .05\) Perfetto
  \item
    \(\leq .08\): Accettabile
  \item
    \(> .08\): No fit
  \end{itemize}
\end{itemize}
\end{frame}

\hypertarget{stima-dei-parametri}{%
\section{Stima dei parametri}\label{stima-dei-parametri}}

\hypertarget{massima-verosimiglianza}{%
\subsection{Massima verosimiglianza}\label{massima-verosimiglianza}}

\begin{frame}{}
\protect\hypertarget{section-4}{}
\textcolor{highlight}{Massima verosimiglianza}
(\textcolor{highlight}{Maximum Likelihood}, \textcolor{highlight}{ML})
\(\rightarrow\) trova i valori dei parametri che massimizzano la
probabilità di ottenere i dati osservati (i.e., i valori che
massimizzano la funzione di verosimiglianza dei dati osservati)

Due tipologie:

\begin{enumerate}
\item
  \textcolor{highlight}{\textbf{Massima verosimiglianza congiunta}}
  (Joint Maximum Likelihood, JML): Permette di stimare per massima
  verosimiglianza \textbf{congiuntamente} i parametri degli item e delle
  persone
\item
  \textcolor{highlight}{\textbf{Massima verosimiglianza marginale}}
  (Marginal Maximum Likelihood, MML): Permette di stimare per massima
  verosimiglianza i parametri degli item, i parametri delle persone sono
  stimati successivamente con procedure bayesiane
\end{enumerate}
\end{frame}

\begin{frame}{}
\protect\hypertarget{section-5}{}
\begin{exampleblock}{Vantaggi ML}

\vspace{3mm}
All'aumentare dell'ampiezza campionaria $\rightarrow$ le stime per ML covergono al valore vero (\textcolor{highlight}{unbiased})

\end{exampleblock}

\begin{alertblock}{Svantaggi ML}

\vspace{3mm}
Le stime di soggetti/item con punteggi estremi sono $+$ o $-$ $\infty$

Per ovviare a questo problema: 

$\rightarrow$ Sottrarre $.30$ se il punteggio è massimo

$\rightarrow$ Aggiungere $.30$ se il punteggio è minimo

\end{alertblock}

\begin{center}
Puteggi estremi:
\end{center}

\begin{columns}[T]
\begin{column}{0.48\textwidth}
Soggetti che hanno dato solo risposte giuste o solo risposte errate
\end{column}

\begin{column}{0.48\textwidth}
Item che hanno ricevuto solo risposte giuste o solo risposte errate
\end{column}
\end{columns}
\end{frame}

\hypertarget{approccio-bayesiano}{%
\subsection{Approccio Bayesiano}\label{approccio-bayesiano}}

\begin{frame}{}
\protect\hypertarget{section-6}{}
Moltiplicando la funzione di verosimiglianza per una distribuzione a
priori si ottiene la \textbf{distribuzione a posteriori}

Le stime dei parametri si ottengono dalla distribuzione a posteriori
\(\rightarrow\) \textcolor{highlight}{stime bayesiane}

Due tipologie:

\begin{itemize}
\item
  \textcolor{highlight}{\textbf{Maximum a Posteriori}} (MAP): La stima
  del parametro è la moda della distribuzione a posteriori di quel
  parametro
\item
  \textcolor{highlight}{\textbf{Expected a Posteriori}} (EAP): La stima
  del parametro è la media della distribuzione a posteriori di quel
  parametro
\end{itemize}
\end{frame}

\begin{frame}{}
\protect\hypertarget{section-7}{}
\begin{exampleblock}{Vantaggi approcci bayesiani}

\vspace{3mm}
Permettono la stima \textbf{finita} dei parametri dei soggetti e degli item anche in caso di punteggi estremi

\end{exampleblock}
\end{frame}

\end{document}
